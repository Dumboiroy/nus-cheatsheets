\documentclass[10pt,landscape,a4paper]{article}
%\usepackage[utf8]{inputenc}
%\usepackage[ngerman]{babel}
\usepackage[normalem]{ulem}
\usepackage{tikz}
\usetikzlibrary{shapes,positioning,arrows,fit,calc,graphs,graphs.standard}
\usepackage[nosf]{kpfonts}
\usepackage[t1]{sourcesanspro}
%\usepackage[lf]{MyriadPro}
%\usepackage[lf,minionint]{MinionPro}
\usepackage{multicol}
\usepackage{wrapfig}
\usepackage[top=0mm,bottom=1mm,left=0mm,right=1mm]{geometry}
\usepackage[framemethod=tikz]{mdframed}
\usepackage{microtype}
%\usepackage{physics}
\usepackage{tabularx}
\usepackage{hhline}
\usepackage{makecell}
\usepackage{mathtools}
\usepackage{amssymb}

\usepackage{listings}

\DeclarePairedDelimiter{\ceil}{\lceil}{\rceil}

\newcommand\codeblue[1]{\textcolor{blue}{\code{#1}}}

\usepackage{lastpage}
\usepackage{datetime}
\yyyymmdddate
\renewcommand{\dateseparator}{-}
\let\bar\overline

\definecolor{myblue}{cmyk}{1,.72,0,.38}

\def\firstcircle{(0,0) circle (1.5cm)}
\def\secondcircle{(0:2cm) circle (1.5cm)}

\colorlet{circle edge}{myblue}
\colorlet{circle area}{myblue!5}

\tikzset{filled/.style={fill=circle area, draw=circle edge, thick},
outline/.style={draw=circle edge, thick}}

\pgfdeclarelayer{background}
\pgfsetlayers{background,main}

%\everymath\expandafter{\the\everymath \color{myblue}}
%\everydisplay\expandafter{\the\everydisplay \color{myblue}}


\renewcommand{\baselinestretch}{.8}
\pagestyle{empty}

\global\mdfdefinestyle{header}{%
  linecolor=gray,linewidth=1pt,%
  leftmargin=0mm,rightmargin=0mm,skipbelow=0mm,skipabove=0mm,
}

\newcommand{\header}{
  \begin{mdframed}[style=header]
    \footnotesize
    \sffamily
    ACC1701XA Midterms Cheatsheet v1.1 (\today)\\
    by~Your Name,~page~\thepage~of~\pageref{LastPage}
  \end{mdframed}
}

\let\counterwithout\relax
\let\counterwithin\relax
\usepackage{chngcntr}

\usepackage{verbatim}

\usepackage{etoolbox}
\makeatletter
\preto{\@verbatim}{\topsep=0pt \partopsep=0pt }
\makeatother

\counterwithin*{equation}{section}
\counterwithin*{equation}{subsection}
\usepackage{enumitem}
\newlist{legal}{enumerate}{10}
\setlist[legal]{label*=\arabic*.,leftmargin=2.5mm}
\setlist[itemize]{leftmargin=3mm}
\setlist[enumerate]{leftmargin=3.5mm}
\setlist{nosep}
\usepackage{minted}

\def\code#1{\texttt{#1}}

\newenvironment{descitemize} % a mixture of description and itemize
{\begin{description}[leftmargin=*,before=\let\makelabel\descitemlabel]}
{\end{description}}

\newcommand{\descitemlabel}[1]{%
  \textbullet\ \textbf{#1}%
}
\makeatletter



\renewcommand{\section}{\@startsection{section}{1}{0mm}%
  {.2ex}%
  {.2ex}%x
{\color{myblue}\sffamily\small\bfseries}}
\renewcommand{\subsection}{\@startsection{subsection}{1}{0mm}%
  {.2ex}%
  {.2ex}%x
{\sffamily\bfseries}}
\renewcommand{\subsubsection}{\@startsection{subsubsection}{1}{0mm}%
  {.2ex}%
  {.2ex}%x
{\rmfamily\bfseries}}

\global\let\tikz@ensure@dollar@catcode=\relax

\def\mathcolor#1#{\@mathcolor{#1}}
\def\@mathcolor#1#2#3{%
  \protect\leavevmode
  \begingroup
  \color#1{#2}#3%
  \endgroup
}

\makeatother
\setlength{\parindent}{0pt}

\setminted{tabsize=2, breaklines}
% Remove belowskip of minted
\setlength\partopsep{-\topsep}

\newcolumntype{a}{>{\hsize=1.5\hsize}X}
% \newcolumntype{b}{>{\hsize=.25\hsize}X}

\setlength\columnsep{1.5pt}
\setlength\columnseprule{0.1pt}

\begin{document}
\setlength{\abovedisplayskip}{0pt}
\setlength{\belowdisplayskip}{0pt}

% \header

\scriptsize
\begin{multicols*}{3}
  \raggedcolumns
  \section{Image Notes}
  \subsection{SOLID Principles}
  \includegraphics[width=\linewidth]{SOLID Principles}
  \subsection{Design Patterns}
  \includegraphics[width=\linewidth]{design-patterns}


  \section{Quiz Mistake List}

  \subsection{Principles}
  \begin{itemize}
    \item SRP $\neq$ 1 method 1 responsiblity. Violation E.g. 1 class adding to DB, sending email, and logging.
    \item OCP - violated when change in behavior requires modifying an existing class instead of extending it.
    Violation E.g. a lot of if-else statements to check for type before executing corresponding code.
    \item Liskov Substitution Principle (LSP) - a subclass must work in place of the super class. 
    No unexpected exceptions, pre/postcondtions, invariants, etc.
    \item Separation of Concerns (SoC) - dont mix logic. Violation E.g. \texttt{AB.add(person); sout("Added person");}
    \item Law of Demeter (LoD): usually violated with train wrecks. Violation E.g. \texttt{b.getSomething().getMoreStuff().doSomething();}
    \item \textbf{SLAP}: Single Level of Abstraction Principle. Violation when many diff levels of indentation.
  \end{itemize}

  \subsection{Design Patterns}
  \begin{itemize}
    \item \textbf{MVC}: View: Displays data, interacts with the user, and pulls data from the model if necessary. | 
    Controller: Detects UI events such as mouse clicks and button pushes, and takes follow up action. Updates/changes the model/view when necessary. | 
    Model: Stores and maintains data. Updates the view if necessary.
    \item \textbf{Singleton Pattern}: can reduce testability.
    \item \textbf{Facade Pattern} increases the amount of code.
  \end{itemize}

  \subsection{Java}
  \begin{itemize}
    \item Constructors of superclasses can be called from subclass. But since constructors aren’t members, they are \textbf{not} inherited.
    \item Polymorphism \textbf{allows} different behaviour if you feed subclasses of the argument class.
    \item Every object has an interface and implementation
    \item Packaging is an aspect of \textbf{encapsulation}
    \item Once a class has 1 or more abstract methods, it is an abstract class.
    \item Unchecked exceptions are not subject to Catch or Specify statements
    \item 2 Classes of Exceptions: Checked \& Unchecked [Runtime exception, Error].
    \item \textbf{JavaDocs} comments can be (1) documentation for developer-as-user and
      for developer-as-maintainer.
  \end{itemize}

  \subsection{Class Diagrams}
  \begin{itemize}
    \item \textbf{+ \& -} are for visibility (not accessibility). 
    [+] public [\#] protected [\textasciitilde] package-private(default) [-] private
    \item \textbf{class-level notation}: underline in class diagram
    \item \textbf{inheritence} Foo → Bar: Foo inherits from bar / Foo extends Bar. Arrow points to super.
    \item Roles should appear at the end of the class that plays the role.
    \item Assoociation Role $\neq$ Association Label
    \item Bot[5] = 0..5
    \item Class should only appear once in a diagram.
    \item \textbf{Association Class}: If dotted line, then don't show variables making it up and vice versa.
    \item \textbf{Association}: can be shown as attributes or association lines \textbf{but not both}.
    \item Mandatory: Class Name, Associations, Multiplicities, 
    \item Possibly Mandatory: Class-level members must be underlined, Attributes(if relevant), Methods(if relevant)
    \item Optional: Visibility, Attribute/Return/Parameter Types, Navigability Arrows
    \item \textbf{Composition $\blacklozenge$} (Whole$\blacklozenge-$Part, \textbf{solid} diamond) Implies (1) delete cascading (2) no cyclical links
    \item \textbf{Aggregation $\Diamond$} (Container$\Diamond-$Item, \textbf{Hollow} diamond)
    \item Observer --> Subject. Arrow points from observer to Subject - Observer depends on Subject.
  \end{itemize}

  \subsection{Object Diagrams}
  \begin{itemize}
    \item Labels must be underlined.
    \item Navigability of associations (arrows in associations) in Object Diagrams are optional. 
    Either direction works, just keep the association role consistent.
  \end{itemize}

  \subsection{Activity Diagrams}
  \begin{itemize}
    \item Optional: Activation Bars | Return Arrows | 
    Object creation/deletion Markers | Message labels (names, parameters, return values)
    \item Mandatory: Lifelines | Messages (solid arrows) | Top to Bottom
    \item Object Names should not be underlined
    \item Activation bar of method should not be broken in the middle (when it calls another method)
    \item Incoming/Return arrows must be at the very top/bottom
  \end{itemize}

  \subsection{Conceptual Class Diagrams}
  \begin{itemize}
    \item Should be used in place of CDs for \textbf{domain modelling}.
    \item \textbf{CCD}: Conceptual Class Diagrams (lighter version of class diagrams) capture class structures in the problem domain, 
      AKA OO Domain Models (OODMs).
    \item \textbf{CCDs} do not show methods or navigability.
  \end{itemize}

  \subsection{Sequence Diagrams}
  \begin{itemize}
    \item Loop segments can happen 0 times.
  \end{itemize}

  \subsection{Project Stuff: Requirements \& Code Requirements}
  \begin{itemize}
    \item \textbf{Brownfield vs Greenfield}: Brownfield harder to modify, Greenfield harder to design. 
      Overall brownfield slightly harder.
    \item \textbf{User stories} are considered light-weight. Recc: Limit to 1 sentence.
    \item \textbf{Magic literals} Try to avoid, but loop counters or simple indexing is fine (not considered magic).
    \item \textbf{KISS}: Keep it simple, stupid - Clever solution only if the 
    additional complexity is justifyable.
    \item Bug Fixing is not refactoring
    \item \textbf{Asserts}: Java assert $\neq$ JUnit(or other) assert. 
    Java asserts are disabled by default at runtime, and are mainly used for
    internal consistency checks.
    \item \textbf{Abstract -> Interface}: In Java, abstract classes don't need to implement interface methods. Only concrete classes do.
  \end{itemize}

  \subsection{Logging}
  \begin{itemize}
    \item Logs are typically written to a log file but it is also possible to log information in other ways 
    e.g. into a database or a remote server.
    \item Java has its own default logging mechanism
  \end{itemize}

  \subsection{Use cases, More Project Stuff \& Abstraction}
  \begin{itemize}
    \item \textbf{Use cases} have no strict requirement to have unique IDs, but might be helpful.
    \item \textbf{Use case benefits}: Help uncover functional requirements, provide common understanding 
    between stakeholders, Guide system design and implementation, Support testing
    \item Textbook recommends to \textbf{mix the bottom-up and top-down} approaches
    \item \textbf{Data abstraction}: abstracting away the lower level data items and thinking in terms of 
    bigger entities.
    \item Types of \textbf{abstraction}: Procedural, Data, Control
    \item \textbf{Coupling Cases}: A has access to B's internal structure, A and B depend on the same global variable
    A receives an object of B as a parameter or return value, A inherits from B.
    \item \textbf{Defensive} programming can result in slower code.
  \end{itemize}

  \subsection{Testing}
  \subsubsection{Equivalence Partitioning (EP)}
  \begin{itemize}
    \item Rememember that in Java, \texttt{isPrime(int i);} non-ints aren't valid, 
    so "all nonints" is \textbf{not} a valid EP.
    \item EPs cannot give a \textbf{Neumann-complete} test suite.
    \item EP for \textbf{Day of Month}: [-MAX..0][1..28][29][29][30][31][32..MAX]
    \item (1) both invalid ends (2) 1-28 in all (3) 29 only leap feb (4) 30 OR 31 days in a month
  \end{itemize}
  
  \subsubsection{Types}
  \begin{itemize}
    \item unit - 1 item
    \item integration - work together
    \item regression - retesting to see if regress
    \item system - whole system meets specification?
    \item acceptance - done by end-users to see if meets req.
    \item \textbf{Alpha Testing}: performed by the users, under controlled conditions set by the software development team.
    \item \textbf{Beta Testing}: performed by a selected subset of target users in their natural work setting.
  \end{itemize}
  
  \subsubsection{Approaches}
  \begin{itemize}
    \item \textbf{Scripted testing}: perform testing based on a pre-defined set of test cases.
    \item \textbf{Exploratory testing}: design test cases on-the-fly.
    \item \textbf{Test-Driven Development (TDD)}: evolve functionality and tests in small increments, writing the test before the functional code.
  \end{itemize}

  \subsubsection{Test Coverage}
  \begin{itemize}
    \item \textbf{Function/method coverage}: based on functions executed, e.g., testing 90 out of 100 functions.
    \item \textbf{Statement coverage}: based on the number of lines of code executed.
    \item \textbf{Decision/branch coverage}: based on decision points exercised, e.g., an \texttt{if} statement evaluated to both true and false in separate test cases.
    \item \textbf{Condition coverage}: based on boolean sub-expressions, each evaluated to both true and false with different test cases.
    \item \textbf{Path coverage}: measures coverage based on possible execution paths through the code.
    \item \textbf{Entry/exit coverage}: measures coverage based on possible calls to and exits from operations in the SUT.
  \end{itemize}

  \subsubsection{Other Notes}
  \begin{itemize}
    \item \textbf{Dogfooding} - Creators using their own product to experience how end users experience it.
    \item \textbf{Unit Testing} isn't just about stubbing, 
    \textbf{Integration Testing} the same as unit testing without stubs.
    \item \textbf{Coverage Analysis} can be useful in improving the quality of testing (not individual test cases).
    \item \textbf{SUT with multiple inputs}: testing all combinations is effective but possible inefficient.
    \item \textbf{\_\_\_ Box Testing} - Knowledge of code: Black box < Grey box < White/Glass/Clear box
    \item \textbf{Path coverage} is much harder to achieve than statement coverage. PC is all possible paths, SC is just code lines.
    \item \textbf{Path coverage} \& Neuman-complete PC - covering all paths is impossile irl since there are loops that may have
    infinite paths. in real life, they check: 0, 1, many, and boundaries.
    IRL tools like Jacoco count branch \& statement coverage but do not attempt path.
  \end{itemize}
  \subsubsection{Efficiency vs Effectiveness}
  \begin{minipage}{0.49\columnwidth}
    \begin{tabularx}{\linewidth}{X}
      \textbf{Efficiency} - How well a system or process achieves its intended goal (doing the right thing)
    \end{tabularx}
  \end{minipage}
  \hfill
  \begin{minipage}{0.49\columnwidth}
    \begin{tabularx}{\linewidth}{X}
      \textbf{Effectiveness} - How well a system or process uses resources while achieving its goal (Doing things well).
    \end{tabularx}
  \end{minipage}
  

  \subsection{Architecture}
  \begin{itemize}
    \item Canvas uses Client-server Architecture style. Client, Server, Database, Communication(Requests)
  \end{itemize}

  \subsection{Random stuff}
  \begin{itemize}
    \item When developing a software to compete with Facebook(or im guessing other large large software), 
    an iterative approach is more suitable than a sequential approach.
  \end{itemize}

  \subsection{Straight up copy paste}
  \subsubsection{Design Patterns in Main}
  \begin{tabular}{l}
    \includegraphics[width=0.7\linewidth]{main-patterns}
  \end{tabular}
  \begin{minipage}{0.15\columnwidth}
    main uses Singleton, Command, Observer. 
  \end{minipage}
  \begin{itemize}
    \item Multiple classes are exposed to the outside, which means it is unlikely the Facade pattern is used.
  \end{itemize}

  \section{Diagram Examples}
  \subsection{Use Case Diagram}
  \includegraphics[width=0.5\linewidth]{UCD}
  \subsection{Activity Diagram}
  \includegraphics[width=\linewidth]{AD}
  \subsection{Conceptual Class Diagram}
  \includegraphics[width=\linewidth]{CCD}
  \subsection{Class Diagram p1}
  \includegraphics[width=\linewidth]{CD-1}
  \subsection{Class Diagram p2}
  \includegraphics[width=\linewidth]{CD-2}
  \subsection{Object Diagram}
  \includegraphics[width=\linewidth]{OD}
  \subsection{Sequence Diagram}
  \includegraphics[width=\linewidth]{SD}
  \subsection{Notes}
  \includegraphics[width=0.5\linewidth]{N&C}
  
  






\end{multicols*}
\end{document}
