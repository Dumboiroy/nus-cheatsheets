\documentclass[10pt,landscape,a4paper]{article}
%\usepackage[utf8]{inputenc}
%\usepackage[ngerman]{babel}
\usepackage[normalem]{ulem}
\usepackage{tikz}
\usetikzlibrary{shapes,positioning,arrows,fit,calc,graphs,graphs.standard}
\usepackage[nosf]{kpfonts}
\usepackage[t1]{sourcesanspro}
%\usepackage[lf]{MyriadPro}
%\usepackage[lf,minionint]{MinionPro}
\usepackage{multicol}
\usepackage{wrapfig}
\usepackage[top=0mm,bottom=1mm,left=0mm,right=1mm]{geometry}
\usepackage[framemethod=tikz]{mdframed}
\usepackage{microtype}
%\usepackage{physics}
\usepackage{tabularx}
\usepackage{hhline}
\usepackage{makecell}
\usepackage{mathtools}

\usepackage{listings}

\DeclarePairedDelimiter{\ceil}{\lceil}{\rceil}

\newcommand\codeblue[1]{\textcolor{blue}{\code{#1}}}

\usepackage{lastpage}
\usepackage{datetime}
\yyyymmdddate
\renewcommand{\dateseparator}{-}
\let\bar\overline

\definecolor{myblue}{cmyk}{1,.72,0,.38}

\def\firstcircle{(0,0) circle (1.5cm)}
\def\secondcircle{(0:2cm) circle (1.5cm)}

\colorlet{circle edge}{myblue}
\colorlet{circle area}{myblue!5}

\tikzset{filled/.style={fill=circle area, draw=circle edge, thick},
outline/.style={draw=circle edge, thick}}

\pgfdeclarelayer{background}
\pgfsetlayers{background,main}

%\everymath\expandafter{\the\everymath \color{myblue}}
%\everydisplay\expandafter{\the\everydisplay \color{myblue}}


\renewcommand{\baselinestretch}{.8}
\pagestyle{empty}

\global\mdfdefinestyle{header}{%
  linecolor=gray,linewidth=1pt,%
  leftmargin=0mm,rightmargin=0mm,skipbelow=0mm,skipabove=0mm,
}

\newcommand{\header}{
  \begin{mdframed}[style=header]
    \footnotesize
    \sffamily
    ACC1701XA Midterms Cheatsheet v1.1 (\today)\\
    by~Your Name,~page~\thepage~of~\pageref{LastPage}
  \end{mdframed}
}

\let\counterwithout\relax
\let\counterwithin\relax
\usepackage{chngcntr}

\usepackage{verbatim}

\usepackage{etoolbox}
\makeatletter
\preto{\@verbatim}{\topsep=0pt \partopsep=0pt }
\makeatother

\counterwithin*{equation}{section}
\counterwithin*{equation}{subsection}
\usepackage{enumitem}
\newlist{legal}{enumerate}{10}
\setlist[legal]{label*=\arabic*.,leftmargin=2.5mm}
\setlist[itemize]{leftmargin=3mm}
\setlist[enumerate]{leftmargin=3.5mm}
\setlist{nosep}
\usepackage{minted}

\def\code#1{\texttt{#1}}

\newenvironment{descitemize} % a mixture of description and itemize
{\begin{description}[leftmargin=*,before=\let\makelabel\descitemlabel]}
{\end{description}}

\newcommand{\descitemlabel}[1]{%
  \textbullet\ \textbf{#1}%
}
\makeatletter



\renewcommand{\section}{\@startsection{section}{1}{0mm}%
  {.2ex}%
  {.2ex}%x
{\color{myblue}\sffamily\small\bfseries}}
\renewcommand{\subsection}{\@startsection{subsection}{1}{0mm}%
  {.2ex}%
  {.2ex}%x
{\sffamily\bfseries}}
\renewcommand{\subsubsection}{\@startsection{subsubsection}{1}{0mm}%
  {.2ex}%
  {.2ex}%x
{\rmfamily\bfseries}}

\global\let\tikz@ensure@dollar@catcode=\relax

\def\mathcolor#1#{\@mathcolor{#1}}
\def\@mathcolor#1#2#3{%
  \protect\leavevmode
  \begingroup
  \color#1{#2}#3%
  \endgroup
}

\makeatother
\setlength{\parindent}{0pt}

\setminted{tabsize=2, breaklines}
% Remove belowskip of minted
\setlength\partopsep{-\topsep}

\newcolumntype{a}{>{\hsize=1.5\hsize}X}
% \newcolumntype{b}{>{\hsize=.25\hsize}X}

\setlength\columnsep{1.5pt}
\setlength\columnseprule{0.1pt}

\begin{document}
\setlength{\abovedisplayskip}{0pt}
\setlength{\belowdisplayskip}{0pt}

% \header

\scriptsize
\begin{multicols*}{4}
  \raggedcolumns

  \section{Notes to myself}
  \begin{enumerate}
    \item Read whole question first. A lot of info is useless. 
    \item For multiple part questions, highlight the key point they are asking for.
    \item Before calculating, write down all equations being used.
  \end{enumerate}

  \section{Accounting in Business \& FS Overview}
  \subsection{International Accounting Standards Board (IASB)}
  \subsubsection{}
  \begin{itemize}
    \item Qualitative Characteristics: Relevance, Faithful Representation,
    \item Comparability, Veriiability, Timeliness, Understandability
    \item IASB -> IFRS -> FRS (Singapore)
    \item Overall ethical conduct - Independent accountants represent the public interest
  \end{itemize}
  \subsubsection{Auditors \& External Audit}
  \begin{itemize}
    \item Independent certified public accountant (CPA) perform external audits.
    \item Provides public with assurance that FSs are not misleading.
  \end{itemize}
  \subsection{The Accounting Equation}
  \begin{itemize}
    \item Assets = Liabilities + Equity
    \item = Liabilities + Share Capital + Retained Earnings
    \item = Liabilities + Share Capital + Revenue - Expenses - Dividends
  \end{itemize}
  \subsubsection{Asset}
  \begin{itemize}
    \item A \textbf{present resource}
    \item Due to a \textbf{past event}
    \item That will \textbf{result in future benefits}
  \end{itemize}
  \subsubsection{Liabilitity}
  \begin{itemize}
    \item An \textbf{obligation}
    \item Due to a \textbf{past event}
    \item That will \textbf{result in future outflow} of resources upon \textbf{settlement}
    \item Accounts: Unearned Revenue, Accounts Payable
  \end{itemize}
  \subsubsection{Equity}
  \begin{itemize}
    \item Owner's claim on the residual interest after deducting all Liabilities
    \item net assets (total assets - total liabilities)
  \end{itemize}
  \subsubsection{Claims}
  \begin{itemize}
    \item Share Capital / Capital Stock: contributed by owners
    \item Retained earnings: equity earned by company
  \end{itemize}
  \subsection{IASB Definitions}
  \begin{itemize}
    \item Income: inflow/enhancement of assets (OR decrease in liabilities -> equity up). During accounting period, not by contributions from owners.
    \item Expense: outflow/depletions of assets (OR incurrance of liability -> equity down). During accouting period.
    \item Capital Maintenance Adjustments: Revaulation of assets/liabilities. \textbf{not included in income statement}. Treated as Other Comprehensive Income (OCI)
  \end{itemize}
  \subsection{The Financial Statements}
  \subsubsection{Statement of Financial Position AKA Balance Sheet (SFP/BS)}
  \begin{itemize}
    \item A = L + E
    \item A snapshot of company's economic resources and obligations
    \item Limitation: Assets recorded at cost, not market value. Usually book v < market v
  \end{itemize}
  \subsubsection{Statement of Comprehensive Income (SCI)}
  \begin{itemize}
    \item \textbf{Net Income (NI)} = Rev - Exp
    \item NI + OCI = Comprehensive Income 
    \item Revenues: Operation earnings, Net Income: NI = Rev - Exp, Other Comprehensive Income: investments etc.
    \item Shows economic performance over time.
    \item Either Income statement \& SCI together or not. OCI can be seperate (IAS1)
  \end{itemize}
  \subsubsection{Statement of Changes in Equity (SCE)}
  \begin{itemize}
    \item How the ownership interest in a company has changed over a specific period
    \item Beginning Equity + $\triangle \text{Equity}$ = End Equity, 
    where $\triangle \text{Equity}$ = $\triangle \text{Capital}$ + Net Income - Dividends + OCI
  \end{itemize}
  \subsubsection{Statement of Cash Flows (SCF)}
  \begin{itemize}
    \item How cash is generated and used by the company
    \item CFO + CFI + CFF (operating, Investing, financing)
    \item CFO: revenue, etc. CFI: buying equipment, land, etc. CFF: investors investing, paying dividends etc.
  \end{itemize}
  \subsection{Relationships among the 4 FSs}
  \begin{tabular}{l}
    \includegraphics[width=0.95\linewidth]{relationship-of-fs}
  \end{tabular}
  \subsection{Fundamental Concepts \& Assumptions of Accounting}
  \begin{itemize}
    \item Separate entity concept: Activity of a bz is separate from its owners
    \item Time-period assumption: bz's activities can be divided into time periods (monthly,quarterly,etc)
    \item Assumption of arm's-length transactions
    \item Cost Principle
    \item Fair value Principle
    \item Monetary measurement concept
    \item Going concern assumption
  \end{itemize}
  \section{Mechanics of Accounting}
  \subsection{Credit/Debit}
  \begin{itemize}
    \item Depends on Type of account.
    \item Normal Debit/Normal Credit
  \end{itemize}
  \begin{tabular}{l}
    \includegraphics[width=0.6\linewidth]{dr-cr}
  \end{tabular}
  \begin{minipage}{0.3\columnwidth}
    depends on where the account lies on the AE.
  \end{minipage}
  

  % legit cba, just gna put in formulas

  \section{The Accounting Equation}
  \subsection{Main equation}
  \begin{itemize}
    \item Assets = Liabilities + Equity
    \item = Liabilities + Share Capital + Retained Earnings
    \item = Liabilities + Share Capital + Revenue - Expenses - Dividends
  \end{itemize}
  \subsection{Assets}
  \begin{itemize}
    \item Current and Non-current
    \item Current: Cash \& Cash eq, Accounts Receivable, Inventory, Prepaid expenses
    \item Non-current: PPE, Intangible assets: trademarks, copyrights, goodwill, 
    Long-term investments: Bonds
  \end{itemize}

  \section{AJE (Adjusting Entries) \& effect on Accounting Equation}

  \subsection{Cash vs Accrual}
  \begin{itemize}
    \item Cash: add up cash inflows and outflows. 
    Revenue recorded as soon as cash comes in, expense recorded as soon as cash goes out. 
    None of: AR, AP, UR, Prepaid Expenses, Accumulated Depreciation  
    \item Accrual: Recognizes event when main economic impact occurs. 
    Revenues recorded when earned, expenses recorded when incurred.
  \end{itemize}

  \subsection{Calculating Effects on AE}
  \begin{itemize}
    \item Supplies worth \$500 was purchased but was not included in assets 
    because it got debited to Supplies Expense even though they were 
    not consumed during the year.\\
    Original | Corrected | effect on accounts | effect on AE
  \end{itemize}
  \begin{tabular}{l}
    \includegraphics[width=0.95\linewidth]{AJE-effect}
  \end{tabular}

  \subsection{Fast way to calculate effect on AE}
  \begin{itemize}
    \item If need \textbf{fast}: (1) \textbf{Focus on variable (A/L/E or others)}
    (2) Correcting Change to Accounts
    (3) effect on variable (A/L/E or NI, Cur Asset, or others.)
  \end{itemize}

  \subsection{Careful of Contra Accounts}
  \begin{itemize}
    \item \textbf{be careful of contra accounts}: category[Contra account(account)]
    \item xAssets[Accumulated Depreciation (PPE), Loss Allowance(AR), Discount on NR(NR)]
    \item xRevenues[Sales returns, Sales Discounts(Sales Revenue)]
    \item xEquity[Treasury Stock(Shareholders Equity), Owner's Draws(Owner's Capital)]
    \item xLiability[Discount on bonds payable(bonds payable)]
  \end{itemize}


  
  \section{Notes Receivables}
   A promissory note to pay a specified amount of money, usually with interest, either
   on demand or at a definite future date.
    e.g. 800k, 4\%, 9 month note on July 1st 2023. \\
    (1) at \textbf{EOY 2023}: Accrued $\frac{6\text{(july-dec2023)}}{12} \times 4\% \times 800k$ worth of Interest Revenue. 
    -> Dr Interest Receivable, Cr Interest Revenue.\\ 
    (2) at March 31st (\textbf{due date}): Dr Cash 800k + full interest (24k), Cr Interest Receivable 16k, 
    Cr Interest Revenue $\frac{3\text{(jan-mar2024)}}{12} \times 4\% \times 800k$,
    Cr 800k Accounts Receivable.

  
  \section{Estimated Credit Loss (ECL) \& Loss Allowance (LA)}
  \begin{itemize}
    \item $ECL = LA_{target} - LA{current}$ where $LA_{target}$ depends on analysis
    \item \textbf{Individual Assessment}: n\% chance that \$x owed by Company A will be uncollectible: 
      $n\% \times \$x = ECL$ (for that company)
    \item \textbf{Group Assesment}: Usually aging analysis: $SUM(\$Amt \times Est.\% Uncollectible)$, 
    \item where each class(current, 1-30 days past due, etc) has some $\$Amt$ collectible and some $Est. \% Uncollectible$.
  \end{itemize}


  \section{Inventory \& COGS}
  \subsection{Inventory}
  \begin{itemize}
    \item NRV > Cost : inventory recorded at cost
    \item NRV < Cost : inventory recorded at NRV
  \end{itemize}


  \subsection{COGS} 
    \begin{itemize}
    \item \textbf{COGS - Cost of Goods Sold}: There's no such thing as inventory expense. 
      [-asset(cash or credit) and +asset(inventory).] You only record it as an expense when 
      you sell the inventory (COGS).
    \item $\text{COGS} = \text{Beg Inv} + \text{Purch'd Inv} - \text{End Inv}$
    \end{itemize}
  

  \subsection{Formulas}
    \begin{itemize}
      \item Weighted Average Cost = total cost/total n
      \item Specific Identification Cost = n*cost per purchasing transaction
    \end{itemize}

  \section{Current Liabilities}
    \subsection{Warrenty}
      \begin{itemize}
        \item During expected warrenty expense calculation: Dr Warrenty Exp, Cr Est. Warrenty Liability
        \item When warrenty claimed: Dr Est. Warrenty Liability, Cr Cash/Inventory for repairs
      \end{itemize}

	\section{PPE \& Intangibles}
  \begin{itemize}
  \item ...
  \end{itemize}

  \subsection{Depreciation Methods}
  \subsubsection{Why Depreciation?}
  Pressure to lower taxable income v.s. Pressure to inflate reported profitability
  \textbf{Remember to calculate partial year if needed.}
  \subsubsection{Straight-line}
  \begin{itemize}
  \item Equal depreciation each year
  \item $DE = \frac{\text{Cost} - RV}{\text{Useful years}}$
  \end{itemize}
  \subsubsection{Calculating Depreciation}
  \begin{itemize}
    \item Acquisition cost = Invoice + Other add-ons + Sales Tax
    \item Annual depreciation expense under straight line = (Acquisition cost – salvage value) / estimated life
  \end{itemize}

  \subsubsection{Units-of-production}
  \begin{itemize}
  \item Varying amounts of depreciation depending on production that year
  \item $DE = \frac{\text{Cost} - RV}{\text{Life in units of production}} \times \text{Actual units produced that year}$
  \end{itemize}

  \subsubsection{Declining-balance/Accelerated Depreciation}
  \begin{itemize}
  \item More depreciation in earlier stage (2x, 1.5x)
  \item (1) Useful life is 4 years -> straight line rate = $\frac{100\%}{4} = 25\$$
  \item (2) Assume double declining -> DDB rate = $2\times25\% = 50 \%$
  \item (3) Assume net book value of asset is \$10,000 -> depreciation expense = \$10k x 50\% = \$5,000 (for that period)
  \item \textbf{Residual value} is \textbf{ignored}.
  \item Once asset is depreciated below RV: $\text{Depreciation Expense} = NBV_{previous} - RV$
  \end{itemize}
  \includegraphics[width=0.95\linewidth]{Depreciation Formulas}

  \subsection{Capitalize v.s. Expense}
  \begin{itemize}
    \item R\&D: Expensed. Dev cost after tech feasiility established can be capitalized. (IFRS) 
    (GAAP: all r\&d are expensed in the period incurred).
    \item Repairs:
    \item Expense: Maintainence, Does not increase productivity, does not extend life beyond original estimate, Recurring in nature.
    \item Capitalize: Overhauls or partial replacements, not frequent, increases efficiency, Extends useful life beyong estimate, Involves a lot of money
    \item NBV before capitalization = Original Acquisition cost – Accumulated Depr
    \item NBV after capitalization = Revised asset value – salvage value 
    \item = (Beg Carrying amount + capitalized exp) – salvage
  \end{itemize}

  \subsection{Disposal of PPE}
  \begin{itemize}
    \item Go through slides again (Slide 47 + post lecture stuff)
    \item PPE Truck + (Gain) = Disposal Value + Accumulated Depreciation + (Loss)
  \end{itemize}

  \subsection{Intagible Assets}
  \begin{itemize}
    \item Definite vs indefinite life
    \item Amortisation expense
    \item Accumulated Amortization (new xAsset)
    \item Market Cap - Net Book value = Goodwill (2638B - 51B for apple lol)
  \end{itemize}

  \subsection{new formulas}
  \begin{itemize}
    \item $\text{Fixed Assets Turnover} = \frac{Net Sales}{Avg Fixed Assets}$
    \item how efficient a company is in using its fixed assets to generate sales
    \item $\text{Total Assets Turnover} = \frac{Net Sales}{Avg Total Assets}$
    \item ability in using its total assets as a whole to generate sales
  \end{itemize}

  \subsection{slide 62 for summary}

  \section{Statement of Cash Flows}
  \subsection{Cash Flow from Operating Activities (CFO)}
  \begin{itemize}
    \item NI = Net Income, $\delta$ CA = $\delta$ Current Assets, $\delta$ CL = $\delta$ Current Liabilities
    \item CFO = NI + Non-cash expenses + Losses - Gains - $\delta$ CA + $\delta$ CL
  \end{itemize}

  \subsection{Cash Flow from Investing Activities (CFI)}
  \begin{itemize}
    \item Purchase/Sales of PPE, Purchase/Sale of Investments
    \item CFI = Cash Inflows from Investing - Cash Outflows from Investing
  \end{itemize}

  \subsection{Cash Flow from Financing Activities (CFF)}
  \begin{itemize}
    \item Issuance/Repayment of debt, Issuance/Repurchase of Shares, Dividends Paid
    \item CFF = Cash Inflows from Financing - Cash Outflows from Financing
  \end{itemize}

  \subsection{Net Change in Cash}
  \begin{itemize}
    \item Net Change in Cash = CFO + CFI + CFF
  \end{itemize}
    
  \subsection{Ending Cash}
  \begin{itemize}
    \item Net Change in Cash = CFO + CFI + CFF
  \end{itemize}

  \subsection{Statement Structure}
  


  \section{Uncategorized}
    \subsection{Other}
    \begin{itemize}
      \item \textbf{Retained Earnings} \\ RE = Rev - Exp - Div + \textbf{Previous RE}  
      \item \textbf{Normal Debit or credit?}\\
      Equity = Rev - Exp - Div.
      Revenue is normal credit, Expenses are normal debit, Dividends are normal debit.
      
      \item \textbf{Expenses}\\
      Operating expenses: Rent, Utilities, Salaries, Advertising exp, Fees, \textbf{Expected Credit Loss (ECL)}|
      Non-operating expenses: Interest expense on loans, Taxes expense.|
      \textbf{COGS} (direct cost of producing/obtaining goods sold)
      \item \textbf{Revenue}\\
      Operating revenue: Service/Sales revenue, Rental income (real estate company), Recurring revenue (SaaS company)\\
      Non-operating revenue: Dividend revenue, Gain on sale of assets, ROyalty income, Forex gains, etc.
      \item $\triangle \textbf{NI} \text{(Net Income)}$ = Revenues (all) - Expenses (all). Remember to sum startNI and $\triangle \textbf{NI}$ for endingNI.
      \item $\triangle \textbf{AR} \text{(Accounts Receivable)}$ = Cash Collected - Credit Sales
      \item \textbf{Current Assets}\\
      (1) Current Assets - Loss Allowance (Contra-asset account)
      (2) For all entries, calculate change to Asset.
      Note that Liabilities are not assets, so if you take on liabilities for assets your assets still go up.
      \item \textbf{Cash Collected from customers}\\
      (finalAR-totalAR) + (final Unearned Revenue - startingUR)
      \item \textbf{Journal Entries for Recovering write-offs}: If we recovered \$828 from previously written off accounts, its:
      [Dr AR Cr LA], THEN [Cr AR Dr Cash]. 2 steps, same with:
      \item \textbf{Journal Entries for Selling Goods} [Dr Cash Cr Revenue] then [Dr COGS Cr Inventory]
      \item $\triangle \textbf{Equity}$ = $\triangle \textbf{Share Capital}$ + Net Income - Dividends (Inferred from SCE)
        = $\triangle \textbf{Assets}$ - $\triangle \textbf{Liability}$ Depending on the question, this must be shifted around.
      \item \textbf{Collect on account}: no longer AR. means collected in cash or equivalents.
      \item n-day note issued on day x of some month: $n - \text{days left in month} - \text{days in next month} - ... = \text{day of payment}$
    \end{itemize}

    \section{Other Accounting Equations}
      \subsection{Return on Assets (ROA)}
      Returns on assets is in ratio form as income divided by assets invested.
      (income/assets invested)
      \[
      \text{ROA} = \frac{\text{Net Profit}}{\text{Average total assets}}
      \]
      Where Average total assets = (Start Assets + End Assets )/ 2\\
      Note: Net profit = net income

      \subsection{Debt Ratio}
      Evaluate \textbf{debt risk}. Ability to pay its liabilities using debt ratio. 
      (liabilities/assets), lower = less risk.
      \[
      \text{Debt Ratio} = \frac{\text{Total Liabilities}}{\text{Total Assets}}
      \]

      \subsection{Profit Margin}
      r/s between sales and net profit. Higher = more profit per \$ sale
      \[
      \text{Profit Margin} = \frac{\text{Net Profit}}{\text{Net Sales}}
      \]
      Note: Net Sales = Sales Revenue\\
      A high profit margin is an indicator of future growth.
      
      \subsection{Current Ratio}
      The current ratio of a company gives us a good indication of the company’s 
      ability to pay its debts when they fall due. The current ratio is 
      calculated by dividing current assets by current liabilities. 
      \[
      \text{Current Ratio} = \frac{\text { Current Assets }}{\text { Current 
      Liabilities }}
      \]

      \subsection{Days' Sales Uncollected}
      how much time is likely to pass before we receive cash receipts from credit sales. 
      \[
      \text{Days' Sales Uncollected} = \frac{\text{Accounts Receivable}}{\text{Net Sales}} \times 365
      \]
      
      \subsection{Accounts Receivable Turnover Ratio}
      Measures how often are receivables collected - how many times a year the company 
      converts its average accounts receivables into cash. 
      \[
      \text{Receivable Turnover Ratio} =\frac{\text { Net Sales Revenue }}{\text 
      { Average Net Receivables}}
      \]
      high ratio = faster collection of receivables -> shorter operating cycle ->
      more cash available for running business. Low RTR could indicate that company
      is allowing too much time for customers to pay.
      When calculating a ratio and have income statement item in the numerator and 
      a balance sheet item in denominator, must calculate the avg balance sheet amount. 
      The quickest way is to take (beginning + ending balance)/2.

      If a company offers terms of net 30 on its sales, we should expect turnover of 12.
      This is because over the entire year, the average accounts receivable should be 
      equivalent to roughly 30 days of sales. Hence, Total sales/Avg AR should be 12.
      >12 -> collect >12x its Avg AR per year -> they collect fast -> more cash for 
      running business.
    
    \subsection{Avg Collection Period/Days to Collect}
      Measures how many days on average it takes the company to collect its accounts receivables.
      \[
      \text{Avg Collection Period} =\frac{ 365 }{\text 
      { Accounts Receivable Turnover}}
      \]
      Should be as close to or lower than its offer terms. i.e. net 30 then try to be <=30

    \subsection{Inventory Turnover}
    
    \[
    \text{Inventory Turnover}= \frac{\text{Cost of Goods Sold}}{\text{Average Inventory}}
    \]
      
      How many times a company sells its entire inventory during a period. If inventory varies a lot, average amounts 
      can be computed from interim periods. 
      
      Applied to analyze short-term liquidity and management of inventory. High ratio = more short-term liquidity.
      Low ratio suggest inefficient use of assets, such as company holding more inventory than it needs. High ratio 
      may suggest that inventory is too low. No simple rule other than high ratio is prefered PROVIDED inventory is adequate to
      meet demand. 
    
    \subsection{Days' Sales in Inventory}
    
    \[
    \text{Days' Sales in Inventory} = \frac{\text{Ending Inventory}}{\text{Cost of Goods Sold}} \times 
    365
    \]
    
    To better interpret inventory turnover, many users measure the adequacy of inventory to meet sales 
    demand. Days’ sales in inventory, also called days’ stock on hand, is a ratio that reveals how 
    much inventory is available in terms of the number of days’ sales. It can be interpreted as the 
    number of days one can sell from inventory if no new items are purchased. This ratio is useful in 
    evaluating liquidity of inventory.

    To interpret turnover, we can use this metric to measure how adequate the inventory is to meet sales demand.
    It reveals how much inventory is available in terms of the number of days' sales. Used to evaluate liquidity of inventory.
    Higher = more liquid inventory.
    
    \subsection{Cash Flow on Total Assets}
    
      The Cash Flow on Total Assets ratio is used with profit-based ratios to help assess a company’s performance.  It is calculated as Net cash 
      flow from operating activities divided by Average total assets.
      \[
      \text{Cash Flow on Total Assets} = \frac{\text{Net Cash from Operating Activities}}{\text{Average Total Assets}}
      \]
      This ratio reflects actual cash flows and is not affected by accounting profit recognition and measurement. It can help business decision 
      makers estimate the amount and timing of cash flows when planning and analyzing operating activities.

\section{Statements}
\begin{itemize}
  \item In order of steps:
  \item Income Statement
\end{itemize}
\includegraphics[width=0.95\linewidth]{IS}

\begin{itemize}
  \item Statement of Changes in Equity
\end{itemize}
\includegraphics[width=0.95\linewidth]{SCE}

\begin{itemize}
  \item Statement of Financial Position
\end{itemize}
\includegraphics[width=0.95\linewidth]{SFP}

\begin{itemize}
  \item Statement of Cash Flows
\end{itemize}
\includegraphics[width=0.95\linewidth]{SCF}


\end{multicols*}
\end{document}
